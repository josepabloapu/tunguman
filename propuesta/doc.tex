\documentclass[a4paper]{article}

\usepackage[spanish]{babel}
\usepackage[utf8x]{inputenc}
\usepackage{amsmath}
\usepackage{graphicx}
\usepackage[colorinlistoftodos]{todonotes}
\usepackage{hyperref}

\title{Propuesta: Fut Zoom\&Tagger}
\author{Francisco Mata, Jose Pablo Apú, Juan Carlos Montero}

\begin{document}
\maketitle

\section*{Justificación}

Brindar una solución de problemas a terceros, es un reto que debe ser experimentado. Ya que lo introduce a un escenario más acorde a lo que uno se enfrentará de ahora en adelante. Por lo que el lograr este desafió lo convierte a un trabajo de suma importancia. Además el colaborar para laboratorio de investigación hace que el proyecto sea más enriquecedor. 

\section*{Objetivo General}
Crear una applicación que logre satisfacer las necesidades del PRISLab.

\section*{Objetivos Específicos}
\begin{itemize}
\item Implementar una función zoom, para que sea más fácil su manipulación.
\item Que el programa sea capáz de etiquetar al jugador en escena y una jugada en determinano intervalo de tiempo.
\item Que el programa maneje una base de datos y guarde los datos en la misma.
\end{itemize}


\section*{Metodología y Cronograma}
\begin{itemize}
\item Diseñar la interfáz gráfica así como sus funciones para tener masomenos la idea de lo se va a crear. \\(semana del 28 al 3 de noviembre)
\item Crear el ambiente de trabajo, es decir instalar todas las librerías necesarias del projecto y tenerla bien configuradas. \\(semana del 4 al 10 de noviembre)
\item Crear los métodos suficientes para cumplir con los objetivos. \\(2 semanas: del 11 al 23 de noviembre)
\item Revisión, para corroborar si cumple con las necesidades del PRISLab. \\(semana del 25 al 1 de diciembre)
\item Corrección y ajusted propuestos por el PRISLab. \\(se espera cuncluir antes del 15 de diciembre)
\end{itemize}

\begin{thebibliography}{10}
\bibitem{tangent}\url{http://qt-projectorg}
\bibitem{opengl}\url{http://gstreamer.freedesktop.org}
\bibitem{dundas}\url{http://gstreamer.freedesktop.org/data/doc/gstreamer/head/qt-gstreamer/html/annotated.html}
\end{thebibliography}


\end{document}
